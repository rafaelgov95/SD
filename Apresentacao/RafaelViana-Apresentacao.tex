\documentclass[12pt]{article}

\usepackage{sbc-template}
\usepackage{graphicx,url}
\usepackage{float}
\usepackage[brazil]{babel}   
%\usepackage[latin1]{inputenc}  
\usepackage[utf8]{inputenc}  
\usepackage{url}

\usepackage{xcolor}
% Definindo novas cores
\definecolor{verde}{rgb}{0.25,0.5,0.35}
\definecolor{jpurple}{rgb}{0.5,0,0.35}
% Configurando layout para mostrar codigos Java
\usepackage{listings}

\bibliographystyle{ieeetr}

\definecolor{lightgray}{rgb}{.9,.9,.9}
\definecolor{darkgray}{rgb}{.4,.4,.4}
\definecolor{purple}{rgb}{0.65, 0.12, 0.82}


\pagestyle{empty}
     
\sloppy
	\title{Prediction-based redundant data elimination with content overhearing in wireless networks  \\ Apresentação I}

\author{Rafael Gonçalves de Oliveira Viana\inst{1} \\\vspace*{10pt} \normalsize  \today{} }


\address{Sistemas de Informação -- Universidade Federal do Mato Grosso do Sul
	(UFMS)\\
  	Caixa Postal 79400-000 -- Coxim -- MS -- Brazil
  \email{rafael.viana@aluno.ufms.br}
}

\begin{document} 

\maketitle

     
\begin{resumo} 	
	Este artigo pretende melhorar a taxa de transferência  de rede sem fio
    por supressão de transmissões de dados duplicados de
	links de rede. Demonstrou-se que a redundância da camada IP
	a eliminação (RE) com o conteúdo excessivo pode significar
	melhorar o bom rendimento ea utilização de canais sem fio em dispositivos sem fio
	meio ambiente. No entanto, a integração da camada IP RE e
	a transmissão sem fio introduz um desafio. Ou seja, probabilístico
	ouvir por acaso sem fio e a possibilidade de um receptor ouvir
	de transmissores múltiplos causam caches de um remetente e um
	receptor longe da sincronização, que pode perturbar a camada de IP
	A correção de RE e degrada seu desempenho.
	Os trabalhos anterios lidam com este desafio pela probabilidade de ouvir por acaso
	que, no entanto, não é eficiente ou escalável.
	Nesse papel, propomos uma eliminação de redundância baseada em previsão com
	método de ouvir por acaso de conteúdo (PRECO) para resolver este desafio.
	Ao explorar o RE baseado em predição, PRECO não requer
	sincronização de cache e estimativa de probabilidade de ouvir por acaso,
	o que permite a sua implantação eficiente e escalável. Baseado em
	PRECO, exploramos os benefícios da implantação do nível de sub-pacote
	RE como um serviço primitivo de camada IP em todos os nós na malha sem fio
	redes, propondo um protocolo de roteamento com reconhecimento de redundância.
	A avaliação de desempenho orientada por rastreamento mostra a eficácia e
	eficiência de PRECO em comparação com outros métodos RE.
\end{resumo}
\section{Introdução}

As redes sem fio têm sido uma comunicação amplamente utilizada
paradigma para proporcionar mobilidade, Internet na cidade
conectividade e computação ao ar livre com baixo custo e rápido
implantação [1], [2], [3], [4], [5], [6]. No entanto, interferência
e a qualidade da ligação fraca limita severamente o rendimento da rede sem fio
redes especialmente para grandes redes densas [7], [8]. este
O papel concentra-se em uma importante classe de técnicas [9], [10],
[11], [12] que visam melhorar o rendimento da rede sem fio
suprimindo transmissões de dados duplicados da rede
links. Essas técnicas podem eliminar as transmissões de
pacotes ou conteúdos de dados que foram previamente transmitidos.
Conseqüentemente, eles podem ser classificados em duas categorias: pacotes baseados
Eliminação de redundância (RE) [9] e baseada em conteúdo
RE [10], [11], [12], [13]. Comparado com RE com pacotes,
RE baseada em conteúdo pode explorar a localidade de dados na carga de trabalho
transferido através de redes sem fio. A localidade dos dados é resultado
do usuário acessa o mesmo conteúdo popular no
Internet e também a similaridade entre diferentes
Objetos de dados da Internet [10].


Estudos recentes mostraram que a grande maioria do tráfego
A redundância na Internet decorre de trocas de dados duplicados de
tamanho inferior a 150 bytes [14]. Por sua vez, o RE de camada IP em um nível de sub-pacote foi mostrado para fornecer informações significativas
benefícios de desempenho em redes com fio. Faz uso de
caches sincronizados entre o remetente e o receptor [15],
[16], [17], [18]. O remetente remove a seqüência de bytes duplicada
(tão pequeno quanto 64B) da carga útil do pacote, comparando-o
contra pacotes previamente transmitidos e, em vez disso, insere uma barra;
o receptor substitui o calço pela seqüência de bytes correspondente
em pacotes recebidos anteriores para reconstruir o pacote completo.

Explorando a técnica RE da camada IP com conteúdo sobre-
ouvir em redes sem fio pode melhorar significativamente a
Goodput e utilização de canais sem fio [12]. Contudo,
Isso apresenta um desafio. Alarme sem probabilidade sem fio
e a possibilidade de um receptor ouvir de múltiplas
os transmissores causam os caches de um remetente e um receptor distante
da sincronização, o que pode interromper a correção da RE
e degradar sua performance. REfactor [12] trata da
desafio pela estimativa de probabilidade de overhearing. Considerando um
infra-estrutura de ponto de acesso único (AP) com múltiplos associados
clientes, o remetente AP estima se o cliente receptor
é provável que tenha ouvido um pedaço de saída do anterior
transmissões para outros clientes. Com base nisso, o remetente
calcula a redução esperada do tempo de transmissão resultante
da remoção do pedaço, em consideração de possíveis
Tempo adicional para solicitação de dados ausentes e retransmissão em
caso de falta de cache no receptor. AP remove apenas o pedaço
quando a redução esperada do tempo de transmissão é alta.

A probabilidade de overhearing sem fio, no entanto, é difícil
estimar com precisão. REfactor determina o problema de overhearing
capacidade apenas pela taxa de transmissão de dados, que é simples
mas muito provavelmente não seja suficientemente preciso por causa da dinâmica
condições de canal e interferência em grandes redes sem fio
redes. Uma estimativa insuficientemente precisa pode causar erros
decisões sobre remoção ou não de redundância, e conseqüências
Reduza o desempenho da RE. Além disso, para estender
para várias infra-estruturas AP e redes de malha, REfac-
O tor exige que os nós sem fio enviem uma auditoria periodicamente
notificações e comunicar conteúdo do cache, que incorre
custo significativo de comunicação e coordenações complexas
entre nós. Da mesma forma, para eliminar a redundância do tráfego
um salto em redes de malha sem fio, um nó deve coletar tal
informações de cache de todos os outros nós dos quais o tráfego
pode ser ouvido pelo nó do próximo salto.


Em vez de usar uma estimativa de probabilidade de
papel aborda o desafio do conteúdo sem fio que ouve demais
para RE de camada IP usando previsões. Consequentemente, propomos
Eliminação de redundância baseada em previsão com conteúdo Over-
audição (PRECO). Em PRECO, o receptor armazena o recebido
e o fluxo de dados ouvido em uma cadeia de pedaços. Ele compara
os pedaços do pacote recebido com as cadeias
cache. Em uma partida, espera-se que a futura entrada
é muito provável que os dados correspondam aos pedaços previamente armazenados
na corrente. O destinatário envia para os futuros dados do remetente
previsões que incluem os hashes de pedaços na corrente. o
o remetente remove o pedaço do pacote de saída se encontrar
que seu hash combina com uma previsão. Desta forma, PRECO
não depende da estimativa de probabilidade de overhearing; ele remove
um pedaço duplicado somente se o receptor tiver armazenado em cache o pedaço.
Ele garante um sistema eficaz e robusto de resposta de conteúdo RE
sobre links sem fio. Também pode ser usado diretamente em múltiplos
Infra-estrutura de AP e redes de malha sem ouvir
notificações e comunicação de conteúdo em cache.



O PRECO tem desafios fundamentalmente diferentes da previ-
A abordagem de RE baseada em Previsão chamada PACK [19] proposta
para o ambiente em nuvem. PACK usa um tamanho grande
8KB o que faz com que ele não consiga identificar o conteúdo de granularidade mais fina
redundância. Também leva a um conteúdo ineficaz que ouve demais
uma vez que muitos nós podem não ouvir tais grandes pedaços em
cheio no ambiente sem fio. Em contraste, o PRECO pretende
identificar pedaços duplicados de centenas de bytes em um sub-pacote
nivelar e trabalhar em links sem fio de baixa largura de banda. Assim, o
O custo de transmissão das previsões deve ser considerado em ordem
para perceber os benefícios gerais da RE. Outra questão nova em
PRECO é a possibilidade de falta de dados em cadeias de partes.
PACK faz previsões e combina não apenas com o recebido
Fluxos TCP, mas também sobre os fluxos de dados ouvidos. Assim sendo,
PRECO deve ser capaz de identificar o fluxo entre os múltiplos
Fluxos contendo partes correspondentes que levam a uma precisão mais precisa
previsão, apesar da falta de dados em alguns dados gerais
fluxos.


Além disso, exploramos os benefícios da implantação de sub-
nível de pacote RE como um serviço de camada IP primitivo em todos os nós
redes de malhas sem fio. Essa implantação em toda a rede pode
fornecer benefícios de desempenho ao eliminar a redundância
Todos os links. Graças ao PRECO que permite uma rede eficiente em toda a rede
implantação em um ambiente sem fio, propomos ainda
roteamento de fonte ciente de redundância em redes de malha sem fio em
obter ganhos de desempenho maiores em toda a rede
RÉ. Apresentamos uma "transmissão estimada consciente de redundância
tempo "(RETT) métrica. RETT prevê a quantidade total de tempo
seria necessário enviar um pacote de dados ao longo de uma rota tomando
em relação ao nível de link RE. Ao encaminhar o tráfego do
Internet para a rede de malha, um gateway escolhe a rota com
o RETT mais baixo.

Em resumo, a contribuição deste trabalho é a seguinte.
\begin{enumerate}
\item PRECO oferece uma solução efetiva, eficiente e escalável
para a camada de IP baseada na camada de IP de conteúdo overhearing over over wireless
links.
\item Um algoritmo de roteamento consciente de redundância é proposto para fur-
explorar os benefícios da implantação de RE em toda a rede em
redes de malhas sem fio.
\end{enumerate}

II revisa os esquemas existentes para a eliminação de redundância em
redes com fio e sem fio. Na seção III e IV, apresentamos 
o design do PRECO e o algoritmo de roteamento com redundância
m, respectivamente. A seção V apresenta avaliação de desempenho.
A seção VI conclui este artigo.




\section{Conclusão}

\bibliography{bibli}

\end{document}


