\documentclass[12pt]{article}

\usepackage{sbc-template}
\usepackage{graphicx,url}
\usepackage{float}
\usepackage[brazil]{babel}   
%\usepackage[latin1]{inputenc}  
\usepackage[utf8]{inputenc}  
\usepackage{url}

\usepackage{xcolor}
% Definindo novas cores
\definecolor{verde}{rgb}{0.25,0.5,0.35}
\definecolor{jpurple}{rgb}{0.5,0,0.35}
% Configurando layout para mostrar codigos Java
\usepackage{listings}

\bibliographystyle{ieeetr}

\definecolor{lightgray}{rgb}{.9,.9,.9}
\definecolor{darkgray}{rgb}{.4,.4,.4}
\definecolor{purple}{rgb}{0.65, 0.12, 0.82}


\pagestyle{empty}
     
\sloppy
	\title{Prediction-based redundant data elimination with content overhearing in wireless networks  \\ Exercício Computacional I - Sistemas Distribuídos}

\author{Rafael Gonçalves de Oliveira Viana\inst{1} \\\vspace*{10pt} \normalsize  \today{} }


\address{Sistemas de Informação -- Universidade Federal do Mato Grosso do Sul
	(UFMS)\\
  	Caixa Postal 79400-000 -- Coxim -- MS -- Brazil
  \email{rafael.viana@aluno.ufms.br}
}

\begin{document} 

\maketitle

     
\begin{resumo} 	
	Este artigo pretende melhorar a taxa de transferência  de rede sem fio
    por supressão de transmissões de dados duplicados de
	links de rede. Demonstrou-se que a redundância da camada IP
	a eliminação (RE) com o conteúdo excessivo pode significar
	melhorar o bom rendimento ea utilização de canais sem fio em dispositivos sem fio
	meio ambiente. No entanto, a integração da camada IP RE e
	a transmissão sem fio introduz um desafio. Ou seja, probabilístico
	ouvir por acaso sem fio e a possibilidade de um receptor ouvir
	de transmissores múltiplos causam caches de um remetente e um
	receptor longe da sincronização, que pode perturbar a camada de IP
	A correção de RE e degrada seu desempenho.
	Os trabalhos anterios lidam com este desafio pela probabilidade de ouvir por acaso
	que, no entanto, não é eficiente ou escalável.
	Nesse papel, propomos uma eliminação de redundância baseada em previsão com
	método de ouvir por acaso de conteúdo (PRECO) para resolver este desafio.
	Ao explorar o RE baseado em predição, PRECO não requer
	sincronização de cache e estimativa de probabilidade de ouvir por acaso,
	o que permite a sua implantação eficiente e escalável. Baseado em
	PRECO, exploramos os benefícios da implantação do nível de sub-pacote
	RE como um serviço primitivo de camada IP em todos os nós na malha sem fio
	redes, propondo um protocolo de roteamento com reconhecimento de redundância.
	A avaliação de desempenho orientada por rastreamento mostra a eficácia e
	eficiência de PRECO em comparação com outros métodos RE.
\end{resumo}
\section{Introdução}

As redes sem fio têm sido uma comunicação amplamente utilizada
paradigma para proporcionar mobilidade, Internet na cidade
conectividade e computação ao ar livre com baixo custo e rápido
implantação [1], [2], [3], [4], [5], [6]. No entanto, interferência
e a qualidade da ligação fraca limita severamente o rendimento da rede sem fio
redes especialmente para grandes redes densas [7], [8]. este
O papel concentra-se em uma importante classe de técnicas [9], [10],
[11], [12] que visam melhorar o rendimento da rede sem fio
suprimindo transmissões de dados duplicados da rede
links. Essas técnicas podem eliminar as transmissões de
pacotes ou conteúdos de dados que foram previamente transmitidos.
Conseqüentemente, eles podem ser classificados em duas categorias: pacotes baseados
Eliminação de redundância (RE) [9] e baseada em conteúdo
RE [10], [11], [12], [13]. Comparado com RE com pacotes,
RE baseada em conteúdo pode explorar a localidade de dados na carga de trabalho
transferido através de redes sem fio. A localidade dos dados é resultado
do usuário acessa o mesmo conteúdo popular no
Internet e também a similaridade entre diferentes
Objetos de dados da Internet [10].
\section{Conclusão}
Neste relatório foi apresentado a estrutura básica de um \textit{Web Services}, e como o pŕoprio pode ser utilizado para consumir serviços de outros \textit{Web Services} disponíveis por outras empresas, outra abordagem exposta seria utilizar os mesmos para criação de um novo serviço, em seu próprio \textit{Web Server}, para atender as mais diversas necessidades, como exemplo o rastreiamento de encomendas ou obter a cidade e o bairro através de um CEP.	  	

	  \bibliography{bibli}

\end{document}


