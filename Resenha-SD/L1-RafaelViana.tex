\documentclass[12pt]{article}

\usepackage{sbc-template}
\usepackage{graphicx,url}
\usepackage{float}
\usepackage[brazil]{babel}   
%\usepackage[latin1]{inputenc}  
\usepackage[utf8]{inputenc}  
\usepackage{url}

\usepackage{textcomp}

\pagestyle{empty}
     

\sloppy
	\title{On the Evolution of Management Approaches,
		Frameworks and Protocols: A Historical Perspective\\ Resenha I - Sistemas Distribuídos}

\author{Rafael Gonçalves de Oliveira Viana\inst{1} \\\vspace*{10pt} \normalsize  \today{} }


\address{Sistemas de Informação -- Universidade Federal do Mato Grosso do Sul
	(UFMS)\\
  	Caixa Postal 79400-000 -- Coxim -- MS -- Brazil
  \email{rafael.viana@aluno.ufms.br}
}

\begin{document} 

\maketitle

     
\begin{resumo} 	
  Está resenha tem como objetivo relatar os fatos importantes, do artigo intitulado por: \textit{On the Evolution of Management Approaches,
  Frameworks and Protocols: A Historical Perspective}, realizado por George Pavlou.
\end{resumo}



\section{Introdução}
 Pavlou em seu artigo, propõe logo em suas primeiras linhas a análise histórica da evolução das abordagens ligadas ao gerenciamento de rede, sendo elas frameworks e protocolos de gestão, dos últimos 20 anos, propondo uma taxonomia relevante, que a constante busca por melhores abordagens, vem possibilitando. contudo o mesmo propõe a análise detralhada da evolução que essas abordagens tomaram.
\subsection{Análise}
 Apoś o autor apresentar seus objetivos, o mesmo começa a fazer uma contextualização das tecnologias estudadas, e dividi assim os principais aspectos de gestão de rede, em 3 tópicos sendo eles;

	 \begin{enumerate}
	 	\item Arquiteutra
	 	\item Modelagem encima do recurso utilizado.
	 	\item Lógica de resolução de problemas e os algoritmos de aplicativos de gerenciamento.
	 \end{enumerate}
 
	Pavlou utilizou 6 seções para formular sua proposta, e relatar suas considerações futuras, a estrutura ficou dividida em; 
  \begin{enumerate}
 	\item Intodução;
 	\item Analise e Evolução da tenologia;
 	\item Apresentação dos modelos gerente-agente, interface objeto/serviço distriubido;
 	\item Aspectos fundamentais das principais tecnologias, incluindo OSI-SM, SNMP, Common Object Request Broker Architecture (CORBA) e Web Services;
 	\item Discute o trabalho restrito;
 	\item Resumo e perspectivas para o futuro.
 	
 	
 \end{enumerate}

Em sua contestualização da evoluçãa da tenlogia, Pavlou segue uma linha do tempo, passando pela primeira abordagem por orientação a objeto, ao nascimento do protocolo leve e facil implementação o SNMP, que após alguns anos se mostrou ineficiente, sendo abandonado pela IETF, e substituido por modelos conhecidos por PIBs \textit{Policy Information Base}.

Em sua introdução da tecnologia CORBA podemos observar que a falta de recuperção de dados pela mesma, foi seu caocanhar de aquiris, tornando-se inviavel para evolução da mesmo, e sendo subistituida por abordagens como o Java Remote Method Invocation (JRMI), que também foi desafiada pela tecnologia emergente dos Serviços da Web.

O autor segue o racicionio da sobrepossição de tenologia com o passar do tempo e aborda os códigos moveis, que era uma abordagem totalmente diferente, iniciada no início dos anos 90.

O autor finaliza seu levantamento de tecnologia falando das abordagens baseadas em XML e na Web, primeiramente apresenta a primira versão do XML em 90, e após suas considerações apresenta a segunda parte da evolução do XML em 2000 pela IETF, para utilização do NetConf, nesta mesma data Web Services utilizavam o idioma XML para comunicação via SOAP ou HTTP.      
\section{M}
	Nos objetivos 
	\subsection{Teste}
	
\section*{Conclusão}
Neste relatório foi apresentado a estrutura básica de um \textit{Web Service}, e como o pŕoprio pode ser utilizado para consumir serviços de outros \textit{Web Services} disponíveis por outras empresas, outra abordagem exposta seria utilizar os mesmos para criação de um novo serviço, em seu próprio \textit{Web Server}, para atender as mais diversas necessidades, como exemplo o rastreiamento de encomendas ou obter a cidade e o bairro através de um CEP, a partir de dispositivos diferentes, porém do mesmo \textit{Web Servece}.	  	

	  %\bibliography{bibli}

\end{document}


