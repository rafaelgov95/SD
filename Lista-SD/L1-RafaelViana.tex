\documentclass[12pt]{article}
\usepackage{sbc-template}
\usepackage{graphicx,url}
\usepackage{float}
\usepackage[brazil]{babel}   
%\usepackage[latin1]{inputenc}  
\usepackage[utf8]{inputenc}  
\usepackage{url}
\usepackage{textcomp}
\usepackage{xcolor}
% Definindo novas cores
\definecolor{verde}{rgb}{0.25,0.5,0.35}
\definecolor{jpurple}{rgb}{0.5,0,0.35}
% Configurando layout para mostrar codigos Java
\usepackage{listings}
\bibliographystyle{ieeetr}
\definecolor{lightgray}{rgb}{.9,.9,.9}
\definecolor{darkgray}{rgb}{.4,.4,.4}
\definecolor{purple}{rgb}{0.65, 0.12, 0.82}

\sloppy
	\title{ Lista de Exercício I - Sistemas Distribuídos}

\author{Rafael Gonçalves de Oliveira Viana\inst{1} \\\vspace*{10pt} \normalsize  \today{} }


\address{Sistemas de Informação -- Universidade Federal do Mato Grosso do Sul
	(UFMS)\\
  	Caixa Postal 79400-000 -- Coxim -- MS -- Brazil
  \email{rafael.viana@aluno.ufms.br}
}

\begin{document} 

\maketitle

     
%\begin{resumo} 	
%\end{resumo}



\section{Quais são as propriedades que identificam os Sistemas Distribuídos. Quais são as
	vantagens e desvantagens em relação a sistemas não distribuídos.}
	\begin{enumerate}
		\item As propriedades de um sd são as seguintes:
		\begin{enumerate}
		\item Acesso a recursos	
			\begin{enumerate}
				\item Recursos antes monopolizados podem ser compartilhados entre usuários.
			\end{enumerate}
		\item Transparência da distribuição
			\begin{enumerate}
				\item Capacidade de se anunciar na rede, sendo ocultando ou não sua existência. 
			\end{enumerate}
		\item Abertura
			\begin{enumerate}
				\item Divulgam os servios na rede por IDL
				\item Interoperável
				\item Portável
				\item Extensível: facilidade em configurar e evoluir.
			\end{enumerate}
		\item Escalabilidade
			\begin{enumerate}
				\item Facil integração com outros sistemas.
				\item facil evolução do sistema.
			\end{enumerate}
		\end{enumerate}
		\item Vantagens VS Desvantagens
		\begin{enumerate}
			\item Vantagens
				\begin{enumerate}
					\item Acesso a recurso como poder computacional, informação ou recursos compartilhados como impressoras.
					\item Disponobilidade por replicas.
					\item Portátil,Escalável, Extensível.
					\item Ocultar parcialmente a transparência.
				\end{enumerate}
			\item Desvantagens
				\begin{enumerate}
					\item Ocultar totalmente a transparência, nem sempre é uma boa ideia.
					\item A relação entre desempenho e transparência esta diretamente ligadas.
					\item Segurança fica menos eficênte 
					\item Nivel de complexidade alto.
					
				\end{enumerate}
		\end{enumerate}

	\end{enumerate}

%	  \bibliography{bibli}

\section{Por que pode ser necessário projetar a arquitetura do sistema antes da fase de
	especificação?}
	A especificação está ligado a como executar o serviço através de interfaces, em determinado ambiente arquitetônico, mediante atender o serviço pretendido. 
Quando um projeto tem sua arquitetura definida antes da especificação, fica claro quais definições de interface serão tomadas pelo projeto, removendo algumas definições que a arquitetura pretendida não oferta, antes de chegar na etapa de especificação.



	
	
	
\section{Escreva sobre cada uma das arquiteturas abaixo.}
\begin{enumerate}
\item \textbf{Centrado em dados}:
\begin{enumerate}
	\item Visa o acesso e atualização de repositorio  .
	\item Os componentes se cominicam apenas por meio do repositorio compartilhado.
	\item Se comunicam por conectores como banco de dados ou arquivo diretório.
	\item Podem ser passivos (repossitorios) ou
	\begin{enumerate}
		\item tem a desvantagem  no modelo de dados padronizado, evolução é dificial para outro modelo.
		\item vantegem em grande repositorios de dados.
	\end{enumerate}
	\item Ativos(surgiiu na IA);
		\begin{enumerate}
		\item baseados em agentes, inteligentes 
		\item Podem organizar em sociedade complexas para realiza determinada tarefa1.
		\item um exemplo seria o quadro negro.
		\item quando um especialista tem uma poderação ele adiciona a mesma.
	\end{enumerate}
	\item o modelo pode ser definido unicamente pelo reposiório.
	\item tem a desvantagem de lidar com um modelo unico de sistema,dificil evolucao e custosa, dificil implementar algo eficiênte, politicas de gerenciamento mais dificil de ser criada.
	
\end{enumerate}

\item \textbf{Cliente-servidor}:
\begin{enumerate}
	\item É derivado do estilo em camadas, é separado entre uma maquina sendo o cliente e outra o servidor, essa separação pode ser fisica ou logicamente.
	\item cont
	\item O cliente requisita informações e o Servidor responde as requisições.
	\item Contem as caracteriscas do modelo em cadamadas.
\end{enumerate}

\item \textbf{Em camadas}:
\begin{enumerate}

\item O sistema de camadas é um exemplo de sistema de componentes que interagem, entre si hierarquicamente. Requisições descem as camadas e respostas sobem as camadas.

\item Cada componente 'L' comunicasse somente com L-1.

\item unica desvantagem é atravessar todas camadas par fazer o processamento
\item Baixo acoplamento entre os componentes.
\end{enumerate}
\item \textbf{Call-return}:

	Um exemplo de conector. 
\item \textbf{Pipes and filters}:

	\item O sistema é visto como uma série de transformações complementares
	\item Possui o objetivo de maximizar a reutilização de funcionalidades pequenas e muito bem definidas;
	\item 
\item \textbf{Baseados em eventos}:
	\begin{enumerate}
		\item Componentes: Unidades de processamento de dados que produzem e tratam eventos que acontecem no sistema
		\item Conectores: Elementos que disseminam os eventos entre os componentes do sistema.
		 Não fazem ligação entre componentes ponto-a-ponto.
		 
		\item Restrições
		 \begin{enumerate}
	
		\item A comunicação entre os componentes deve acontecer exclusivamente através dos eventos do sistema
		\item Os conectores apenas comunicam eventos, não executam transformação de dados
			\end{enumerate}
		\item É possível um componente gerar um evento para todos os componentes, ou para alguns. 
		\item Componente publique um evento em um middleware de controle. Somente os componentes que estiverem subscritos no middleware poderão receber a mensagem.
		\item Permite combinar outros estilos arquitetônicos no middleware.
	\end{enumerate}
\item \textbf{ Orientado a agentes}:
	
		
\item \textbf{Máquina virtual}:
\begin{enumerate}
\item Estilo arquitetural também conhecido como máquina abstrata. 
\item Este estilo arquitetural é indicado quando é necessário executar atividades intermediárias, antes da execução da funcionalidade propriamente dita
\item Maquina como java container.
\item desaclopamento 
\end{enumerate}
\end{enumerate}

\section{Analisando os enunciados de sistemas de software apresentados a seguir, recomende um
	estilo arquitetural para cada um deles, justificando a sua opção:}

\begin{enumerate}

	\item Funcionarios processamento de salario
	 
\textbf{		Batch}
	
	\item Sistema de comunicação tipo walkie-talkie para smartphones.
	
\textbf{		per-to-perto ou pipe}
	
	\item Uma loja virtual no ramo de calçados. Em que usuários possam efetuar compras a partir de dispositivos móveis e computadores, de acesso à Internet.
\textbf{	
	cliente servidor}
	
	\item Um sistema de vídeoconferência entre vários usuários localizados em locais distintos. Esse
	sistema deve permitir a exibição de áudio e vídeo (em “tempo real”) para vários participantes ao
	mesmo tempo.
	
\textbf{	Per-to-Per ou MCU}

\item Terminal rio de janeiro

\textbf{} 
\end{enumerate}



\section{Os requisitos abaixo correspondem a um sistema distribuído a ser implementado. Descreva
uma solução arquitetônica que contemple tais requisitos. Justifique sua resposta de acordo com os
requisitos de qualidade. Observação: a solução pode agregar um ou mais estilos arquitetônicos.}

\begin{enumerate}
\item É um sistema WEB;
\item O acesso deve ser feito de maneira segura;
\item A interface gráfica deve ser desacoplada do restante da aplicação;
\item A interface gráfica deve ser facilmente personalizável;
\item  O sistema deve implementar persistência de dados;
\item As regras do negócio devem ser facilmente evoluídas;
\item	 O sistema deve ser escalável, de acordo com o número de acessos simultâneos.
\end{enumerate}

\textbf{Cliente Servidor}
\end{document}

